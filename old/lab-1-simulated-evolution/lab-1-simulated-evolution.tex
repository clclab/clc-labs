\documentclass[a4paper, 9pt]{article}
\usepackage{../styles}
\title{Lab 1 --- Simulated evolution}
\author{Evolution of Language and Music}
\begin{document}

\maketitle

\begin{goals}
In today's computer lab you will experiment with simulated evolution and
look at a simple model of the evolution of communication. The goals of
this lab are:
\begin{itemize}
\item to better understand the concepts of genotype, genotype space, fitness, fitness landscape, selection, mutation, selection-mutation balance, frequency dependent selection;
\item to see how these concepts can be formalised in a computer program;
\item to appreciate both the power and the limits of natural selection.
\end{itemize}
\end{goals}

\paragraph{Prerequisites}
It is highly recommended to use R Studio. You furthermore need the two R files in the \file{materials} folder: \file{auxiliary\_functions.R} and \file{lab-2.R}.
\textbf{Todo: kort uitleggen verschil bestand en console in R Studio}



\section{Simulated Evolution}\label{simulated-evolution}

In the first part of this computer lab, we will use the programming
language \texttt{R} to simulate the evolution of a (DNA) string under a
particular fitness function. First, we will launch the interface for
this programming language, and load the required packages:

\begin{exercise}
    \action Start R-studio (depending on your operating system and preferences).
     \action\textbf{Todo: create new file}
     \action 
     \textbf{Todo: working directory}
     To run a script in \texttt{R}, you type \texttt{source('scriptname')} in the command line. To be able to run a script, it should be available to the interpreter. You should make sure that both the file \texttt{lab-2.R} and the file \texttt{auxiliary\_functions.R} are accessible. If you are working in R-studio, the easiest way to do this, is to put them both in the same folder and set this folder as the \textit{working directory} of R-studio. \href{https://support.rstudio.com/hc/en-us/articles/200711843-Working-Directories-and-Workspaces}{Look at this website} to find out how to do that for your version of R-studio. You can use tab for auto completion.

    \action Install the package stringr by typing \verb|install.packages("stringr")|. As you (probably) have no rights to install the package globally, the computer will ask you if you want to install the package in a personal library, click okay and accept the default settings. If nothing seems to happen, it could be that the pop-up window appeared below another window.
    \action Load the library \texttt{stringr} by typing \texttt{library(stringr)} in the console.
\end{exercise}


We will represent DNA strings (the \textit{genotype}) as a sequence of
the letters `A', `G', `C' and `U'. In \texttt{R}, you can generate a
random sequence of 10 of these letters using the following command:

\begin{lstlisting}
sample(c('A','G','C','U'), size=10, replace=TRUE)
\end{lstlisting}
You can store the output under a name (for instance \texttt{x}), you can
type:
\begin{lstlisting}
x <- sample(c('A','G','C','U'), size=10, replace=TRUE)
\end{lstlisting}
You can then view the contents of a particular object by simply typing
its name and pressing enter. Let's experiment a bit with this.

\begin{exercise}
    \action Using the commands you just learned, generate a few random sequences of length 10 containing the characters A,G,C and U to confirm that it does what we want it to do. (hint: use the up-arrow key to scroll to previous commands).
    \action Generate a random sequence of length 50 containing the characters `A', `G', `C' and `U'. 
    \askstar The set of all possible sequences is called the \textit{genotype space}. How big is this space? I.e., how many genotype strings are possible with our representation?
\end{exercise}

Now, let's create a \textit{population} of DNA strings. To do this, we
will make 100 genotype strings.
We will store our population in a matrix (the \textit{population matrix}), where each member of our population is represented as a row of the matrix. 

\begin{exercise}
\action Let's start with creating a matrix filled with zero's that we can later fill:\footnote{The command \texttt{matrix(x, height, width)} command transforms a vector \texttt{x} into a matrix with height \texttt{height} and width \texttt{width}.} 
  \begin{lstlisting}
population <- matrix(rep(0, 100), 100, 50)}
\end{lstlisting}
  So what do we find in column 30 of our population matrix, and what does this number mean?
  \action Fill your matrix by generating 100 population members in a for-loop and filling the matrix with them:\footnote{\texttt{x[i,]} accesses the \textit{i}th row of the matrix \texttt{x}, which in our case thus corresponds with the \textit{i}th member of our population}
  \begin{lstlisting}
population_size <- 100
for (i in 1:population_size) {
    population[i,] <- sample(c('A','G','C','U'), 
      size=50, replace=TRUE)
}
    \end{lstlisting}
\end{exercise}

Now we need to define a fitness function that computes the fitness of
the individual members of our population. Imagine, for instance, that
the string `CAC' codes for some very useful aminoacid. The more CAC's in
the genome, the higher the expected number of offspring. In our
simulation of evolution, let's define the fitness as the number of times
the substring `CAC' appears in the genotype string (without overlapping, so `CACAC` contains \emph{one} copy).
To keep track of the fitness of \textit{all} members in our population
(which are represented as rows in the population matrix), we create a
\textit{vector} containing where each element of the vector represents
the fitness value for one member of the population.

\begin{exercise}
    \action Generate an empty vector to store the fitness values, and call it \texttt{fitness}:
    \begin{lstlisting}
fitness <- rep(0, population_size)
    \end{lstlisting}
    \action Use a for-loop to fill the vector (created by the code above) with the fitness values:
    \begin{lstlisting}
# loop over population size
for (i in 1:100) {      
  # generate string representation
  member <- paste(population[i,], collapse='') 
  
  # compute fitness member   
  fitness_member <- str_count(member, "CAC")   
  
  # store in fitness vector
  fitness[i] <- fitness_member
}
\end{lstlisting}
    
    (Note that \texttt{R} ignores everything that follows the character \#. In programming terms, these texts are called \textit{comments}.)
    
    What is the highest possible fitness a member of this population can have?
    \action Compute the mean fitness of your population by using \verb|mean(fitness)|.
    What is the average fitness of your population?
\end{exercise}


Now we will generate the next generation. We assume that each member of
the next generation inherits the genome of one of the members of the
previous generation. The probability of inheriting each genome is
proportional to the genome's fitness: a child is most likely to inherit
the genome of the fittest member of the previous population. This
simulates selection.

\begin{exercise}
    \action Compute the average fitness of the population and store it in a variable using:
\begin{lstlisting}
av_fitness <- mean(fitness)
\end{lstlisting}
    \action Generate 100 new children, using the built-in function \texttt{sample} (the same one we used before):\footnote{We first draw 100 random numbers between 1 and 100 (repetitions possible). If population member 2 has a very high fitness, it will have a very high chance of being drawn. Then we use the drawn numbers to create a new population of the members corresponding to the numbers.}
\begin{lstlisting}
indices <- sample(100, size=100, replace=TRUE, 
                  prob=fitness/sum(fitness))
new_population <- population[indices,]
\end{lstlisting}
    \askstar If one population member has fitness 10 and all the other population members have fitness 1, what is the probability that a child will inherit its genome from this one population member? What do you expect to happen with the population?
    \action To simulate the evolution of the population, we want to repeat this process several times and plot the average fitness over time. If you like programming, you can try to do the implementation yourself, but we also provided a script called \texttt{lab-2.R} that does the trick. The next bullet point contains some instructions for implementation, if you use the script you can thus skip them.

    \action To repeat the previous process 100 times, you should create a for-loop that executes the previous bits of code 100 times, storing the fitness of every population in a vector. You can plot your results using
\begin{lstlisting}
plot(seq(1,100,1), av_fitness, type="l", ann=FALSE)
\end{lstlisting}
    (Assuming you stored the fitness values in av\_fitness).
    To label the axes and title of the plot use:
\begin{lstlisting}
title(main="title", xlab="x label", ylab="y label")
\end{lstlisting}
    \askstar You will notice the fitness stops increasing quite early in the simulation. Why is this? (note that lab-2.R will create a new random population matrix)
\end{exercise}


In the previous simulation, we looked at selection \textit{without}
mutation. Let's now look at the case where every child's nucleotide has
a probability $\mu$ to change into a random other nucleotide.

\begin{exercise}
    \action If $\mu=0.01$, what is the chance that no changes occur in a genome. What is the chance that no changes occur in an entire population? And if $\mu=0.001$?
    \action Use the provided script to do the same simulation, but with a mutation level $\mu=0.001$. You can chance the values of the parameters at the top of the script. After changing them, save the file and run the script again by typing \verb|source('lab-2.R')|. Adapt the length of the simulation to a number you think is suitable.
    \action Now repeat the simulation with $\mu=0.01$, plot the fitness. This shows the mutation-selection balance.
    \askstar Why does the fitness with relatively high mutation rate level stop increasing a slightly lower level?
\end{exercise}

\section{Evolution of communication}\label{evolution-of-communication}

In the second part of this assignment, we will model the evolution of a
communication system. A possible way of representing a communication
system is by using matrices that describe a mapping from a set of
meanings to a set of forms (or signals). For instance, the well known
alarm call system of Vervet monkeys \cite{seyfarth1980monkey} in its
usual idealisation, can be described as follows:

\begin{align*}\footnotesize
S &=
  \left(
  \begin{array}{c|ccc}
  & \text{chirp} & \text{grunt} & \text{chutter}\\
  \hline
  \text{leopard} & 0.8 & 0.2 & 0\\
  \text{eagle} & 0.1 & 0.9 & 0\\
  \text{snake} & 0.05 & 0.1 & 0.85 \\
  \end{array}
  \right)
  \\
R &=
  \left(
  \begin{array}{c|ccc}
  & \text{leopard} & \text{eagle} & \text{snake}\\
  \hline
  \text{chirp} & 0.9 & 0 & 0.1\\
  \text{grunt} & 0 & 1 & 0\\
  \text{chutter} & 0.2 & 0 & 0.8\\
  \end{array}
  \right)
\end{align*}

The $S$ matrix represents the sender: the first column contains the
meanings (or situations) that the sender may want to express, the first
row the signals that it can use to express these meanings. The numbers
in the matrix represent the probabilities that the sender will use a
certain signal to express a certain meaning. The matrix $R$ describes
the behaviour of the receiver in a similar way: the numbers in the
matrix are the probabilities that the receiver will interpret a certain
signal (first column) as having a certain meaning (first row).

\begin{exercise}
\askstar What are the optimal $S$ and $R$, for maximal communicative success in a population?
\askstar How is ambiguity (i.e., one signal with multiple meanings) reflected in S and R matrices? And synonymity (two signals that have the same meaning).
\end{exercise}

By using a bit of a trick, we can study the evolution of such a
communication system using the same protocol as in the first part of
this assignment. The $S$ and $R$ matrices of an individual are uniquely
defined by 18 numbers. Assume that we model this by saying that every
individual is characterized by a genome of length 18, where each
nucleotide codes for one value in S and R. Let's say $A=3, G=2, C=1$ and
$U=0$. To construct the S and R matrices, we put the numbers corresponding
to the nucleotides in two matrices and normalise the rows, such that the
probabilities add up to 1.

\begin{exercise}
\ask What would a genome corresponding to the S and R matrix depicted above look like?
\ask Can you think of two strings that have a different genotype but the same phenotype?
\end{exercise}

Of course our previous fitness function --- the count of the substring
``CAC'' --- does not make much sense in this case. We will have to define
a new one. We can compute the chance of successful communication between
two agents by summing up the chance of success for each individual
meaning-signal combination. For instance, let's assume the sender wants
to convey the meaning ``leopard''. We multiply the probabilities for all
signals the sender could use for this meaning (the row in $S$ starting
with ``leopard'') with the chance that the receiver will interpret this
signal as having the meaning ``leopard'' (the ``leopard'' column in
$R$). In this case as the sender only uses the signal chirp to express
the meaning leopard, and the receiver interprets this signal as having
the meaning leopard with probability 1, the chance of successfully
conveying the meaning ``leopard'' in this system is 1. Due to the fact
that agents have both a sender and receiver matrix, it is possible that
the communication in one direction runs flawlessly, but any
communication in the other direction is unsuccesfull. We define the
fitness as the sum of the chances of success for all meanings in both
directions.


\begin{exercise}
 \ask Is it possible to compute the fitness of one individual without taking into account who he is communicating with? Why (not)?
 \end{exercise}

We implemented some fitness functions that you can find in the file
\texttt{auxiliary\_functions.R}:

\begin{itemize}
\item \texttt{CAC\_count}: This is the fitness function you used before, that counts the number of occurrences of the substring "CAC" in the genome;
\item \texttt{communication\_fixed\_target}: This fitness function captures how well the population member can communicate with a fixed target with S and R matrices that allow perfect communication (i.e., this other population member does not use the same signal for different meanings, or assign different meanings to the same signal).
\item \texttt{communication\_random\_target}: This fitness function describes the more realistic situation, in which the fitness of a population member is determined based on its communication with a random other member of the population.
\end{itemize}

\begin{exercise}
\action Load the auxiliary functions library by typing 
\begin{lstlisting}
source('auxiliary_functions.R')
\end{lstlisting}
in the terminal. Leave the file \texttt{auxiliary\_functions.R} untouched, you don't have to change anything there.
\end{exercise}

You can change the fitness function - like the rest of the parameters -
at the top of the file \texttt{lab-2.R}, by uncommenting the line with
the preferred fitness function (and commenting out all other fitness
function lines). As you may have guessed, you can (un)comment a line in
an R script by placing (removing) a `\#' at the beginning.

\begin{exercise}
\ask What is the maximal fitness that an individual can have?
\action Change the fitness function in the script to \texttt{communication\_fixed\_target}. Run an evolutionary simulation with a low mutation rate for 100 iterations. What is the average fitness and most frequent communication system at the end of it? You can check the population at the end by typing \texttt{population} in your command line.
\ask Can the members of the resulting population also communicate with each other or only with the preset fixed target?
\ask What would happen if the target was fixed, but not perfect? You can test your assumption by changing the target matrices in the \texttt{auxiliary\_functions file}. 
\end{exercise}

A more realistic situation is the one in which the members of the
population do not all communicate with the same fixed target, but with
other members of the population, that have their own (evolved) S and R
matrix.

\begin{exercise}
\action Run some evolutionary simulations for this scenario (compute the fitness by using the function \texttt{communication\_random\_target}). What is the average fitness and most frequent communication system at the end of it? Experiment with the learning rate.
\askstar This is frequency dependent selection. Why does it not always evolve to the optimal communication system?
\askstar What do you expect to happen if only successfully \textit{receiving} but not \textit{sending} contributes to fitness?
\action Test your assumption by using the fitness function \texttt{sending\_random\_target}. 
\end{exercise}

\printbibliography

\end{document}

