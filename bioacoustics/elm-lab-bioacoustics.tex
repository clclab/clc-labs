\documentclass[a4paper, 9pt]{article}
\usepackage{../styles}
\title{\textsc{computer lab}\\ \textbf{Bioacoustics in Praat}}
\author{Evolution of Language and Music}
\date{}
\begin{document}
\maketitle
\blfootnote{%
  \emph{Last updated on \today{}.
  Written by Bastiaan van der Weij and Dieuwke Hupkes in 2016.  
  Updated by Peter Dekker (2017), Bas Cornelissen (2017, 2018) and Marianne de Heer Kloots (2018).}}

\begin{goals}
Sounds are the raw materials in the study of language and music. In this
lab we'll learn how to use Praat for analysing and editing sounds. We'll
explore sound signals and look at how they relate to the things we
perceive, such as words, melodies or rhythms.
\end{goals}

\section{Getting started}\label{getting-started}

Praat is a free and open-source computer program widely used in
phonetics (the study of human speech) and bioacoustics. It is a
swiss-army knife containing many tools for visualising, analysing and
synthesizing sounds.

\begin{exercise}
  \action Go to \url{www.praat.org}, find the download page for your favourite operating system and follow the installation instructions until you have started the Praat program (usually this involves double-clicking a beautiful pink icon).
  \action You will see two windows: \textbf{Praat objects} and \textbf{Praat
  picture}. \\
  \textbf{Praat objects} is where the sounds you are editing or
  analyzing will appear. \textbf{Praat picture} is where you can visualize
  the output of various analyses.
\end{exercise}


\section{The anatomy of a sound}\label{the-anatomy-of-a-sound}

From the \textbf{Praat objects} window, navigate to \emph{Open
> Read from file}, or type Ctrl-O (Cmd-O on Mac). In the \file{materials}
folder, you'll find a file called \file{sine.wav}. Open and load it into
Praat. Now that we have a Praat object, let's have a look at what we can
learn. First, let's play the sound.

\begin{exercise}
\action Play the sound by selecting it from \textbf{Praat objects} and clicking \texttt{Play}.
\end{exercise}


\paragraph{Soundwaves and spectrums}\label{soundwaves-and-spectrums}

The waveform is the most straightforward visual representation of a
sound. The waveform is a plot of how the air pressure changes over time.

\begin{exercise}
\action Click \texttt{View \& Edit} to look at the waveform of our sound.
You'll see two visual representations of the sound. The waveform is the
upper one.
\action In the \texttt{View \& Edit} window, zoom in on the waveform until you can clearly see the shape of the sound waves.
\end{exercise}

You'll notice that this sound wave consists of a constantly repeating
pattern. Each repetition of this pattern constitutes one vibration. The
number of vibrations per second is called the \emph{frequency} of the
sound.
Let's try to find out the frequency of the sound we've opened. To do
this, we'll use a different representation of the sound, called a
\emph{Spectrum}. The Spectrum can be stored in a new Praat object (apart
from sounds, Praat objects can also represent other information, such as
the results of various sound-analyses).

\begin{exercise}
\action In the Praat objects window, select \texttt{Analyse Spectrum -}, and click \texttt{To Spectrum...}. Accept the default settings by clicking OK.
\action Select the Spectrum object (if it isn't already selected) and visualize it by clicking the \texttt{View \& Edit} button.
\action Study the window and play around with it for a while: click anywhere in the plot, drag the mouse. What does the x-axis represent? What does the y-axis represent?
\action Find the x-coordinate of the peak in the spectrum as precisely as possible. You'll probably need to zoom in a bit to do this accurately (tip: select the area around the peak you want to study and select ``Zoom to selection'' from the View menu at the top of the window, or press Ctrl-n). What is the frequency of the sound? 
\end{exercise}

As you have heard, and seen, this sound is not particularly exciting.
Let's look at a more interesting sound.

\begin{exercise}
\action Load the file \file{bassoon.wav} into a Praat object 
\action Listen to both sounds (\file{bassoon.wav} and \file{sine.wav}) and compare. Do you hear any similarites, if so which? Which differences do you hear?
\action Open the waveform view and zoom in (to somewhere in the middle of the sound) until you can see the individual vibrations of air-pressure (you can use the same zoom to selection technique that you used previously).
\end{exercise}

\newpage
\noindent You should notice that the individual vibrations form a self-repeating
pattern.

\begin{exercise}
\action Find the shortest pattern in the waveform that contains no repetitons (Praat may already have marked this for you).
\action Place the cursor at the start of the pattern, write down the exact time marking of the cursor. 
\action Place the cursor at the end of the pattern (exactly where it begins to repeat itself again), write down the exact time marking of the cursor. 
\ask Using the two time markings, calculate the frequency (in repetitions per second) of the pattern you found. 
\end{exercise}

The frequency you just found---the frequency of the shortest
non-repeating pattern---is called the \emph{fundamental frequency}. The
fundamental frequency usually (but not always) corresponds to perceived
pitch. As we will see now, sounds often contain many more frequencies,
which can be discovered by looking at the spectrum.

\begin{exercise}
\ask Having analyze the fundamental frequency of \file{bassoon.wav} and frequency of \file{sine.wav}, can you now, more precisely, describe the similarity between the two sounds?
\end{exercise}

\begin{exercise}
\action Create a Spectrum object of \file{bassoon.wav} and display it with View \& Edit. 
\action Can you find a peak in the spectrum corresponding to the frequency you found before?  
\askstar Does the pitch that we perceive (the fundamental frequency) always correspond to the frequency of the highest peak in the spectrum? 
\action Read the frequencies of some other peaks in the spectrum. What do you notice about their relation to each other?
\end{exercise}

The peaks you found in the spectrum are called harmonics. The same note
on various instruments may have the same pitch, but the energy
distribution over the harmonics varies, resulting in different
\emph{timbres}. The same principle allows us to distinguish between
different vowels.


\subsection{The waveform and
spectogram}\label{the-waveform-and-spectogram}

Now we'll look at human vocalizations.

\begin{exercise}
\action Load the files \file{baby-1.wav} and \file{baby-2.wav} into Praat and listen to both sounds.\footnote{During the lecture, you heard cries from a French and a German baby. These were used in a study done by \cite{Mampe2009}. The recordings in this lab were recorded for a recent follow-up study done by \cite{Wermke2016} comparing German and Chinese babies. Have a look at the studies if you're interested, in particular the first one! Both are included in this lab's materials.}
\action Click View \& Edit to look at the waveform for one of the files. Without zooming in, which properties of the sound can you recognize by just looking at the waveform?
\end{exercise}

As you can hear and see, these sounds are more complex than the sounds
we've dealt with so far. The previous two sounds didn't change in pitch
and maintained a (relatively) constant timbre throughout their duration.
In the new sounds, the pattern of vibrations in is continuously
changing. Counting vibrations or looking at the spectrum will not be
able to tell us much. With these sort of sounds, a \emph{spectogram} is
a much more informative visualisation. You can view the spectogram in
the View \& Edit window, just below the waveform. However, we're going
to explore some Praat functionality to draw two spectograms above
eachother in a picture.

We've seen how to view and edit Praat objects. Praat has different
viewers for different objects. In these viewers, you can interact with
the objects and zoom in to regions of interest. However, when you're,
for example, writing a paper, you want to draw nice pictures containing
these visualisations. For this reason, most Praat objects can be drawn
into the \textbf{Praat picture} window. The praat picture, in turn, can
be exported to various image formats.

\begin{exercise}
\action Select one of the two baby sounds.
\action In the \textit{Praat picture} window draw a rectangle with a width of six and height of four (click and drag the mouse).
\action Create a Spectogram object. Click on the ``Analyse spectrum -'' button. From there, click on the ``To spectogram...'' button and accept the default settings.
\action Select Spectogram object that you just created, click ``Paint...'' (under the ``Draw - '' button) and accept the default settings.
\action Draw a second rectangle below the first one. Use the second rectangle to draw the Spectogram of the other baby sound.
\ask Suppose you have heard the two sounds, and are now given these two spectograms. Would you be able to figure which spectogram belongs to which baby sound? If so, how?
\askstar What information does a spectrogram visualize? What do the x- and y- axes represent? What does the darkness of pixels mean?
\end{exercise}

A common analysis used for sounds is the F0 analysis, or fundamental
frequency analysis. As we've seen above, the fundamental frequency
generally corresponds to perceived pitch. We can use Praat to draw a
\emph{pitch contour}.

\begin{exercise}
\action Erase your Praat picture, by going to the Praat picture window, and clicking `Erase all' under the `edit' menu.
\end{exercise}

If you want, you can change the color and thinkness of the drawn lines
to make them stand out better. To do this

\begin{exercise}
\action Open the ``Pen'' menu, and set the line width to 2.0 (by clicking on `Line width...')
\action In the same menu, change the color from black to something else. For example, red. 
\end{exercise}

Now we'll run the F0 analysis and draw the results.

\begin{exercise}
\action Go to the Praat objects window.
\action Select the *sound* object you want to analyze.
\action Under `Analyse periodicity', click `To pitch...'
\action Draw the created pitch object using the same method we used earlier.
\ask How do you think does Praat construct the pitch contour given a sound? Think of the manual analyses we did before. Describe the process informally, i.e., you don't need to be very precise.
\end{exercise}

\section{Speech}\label{speech}

Although we're all very good at producing and interpreting speech
sounds, recognizing sounds in waveforms in spectograms is much harder
(even though they contain the same information!).

\subsection{Phonemes}\label{phonemes}

Phonemes are the basic components of speech. The word ``slit'', for
example consists of a fricative, a liquid, a vowel, and a plosive or
stop (both so called ``consonants''). Plosives are generated by
completely stopping the airflow for a fraction of a section, resulting
in complete silence.

\begin{exercise}
\action Load the file \file{slit.wav}
\action Take a look at the waveform and spectogram and listen to the file
\end{exercise}

By looking carefully at the waveform and spectogram, see if you can
identify the individual phonemes making up the word. This may be harder
than you expect.

\begin{exercise}
\action To verify your identifications, extract each phoneme into a separate Praat object. Select the phoneme in the sound signal (you can either drag in the waveform or in the spectogram), and click ``File'' $>$ ``Extract selected sound (preserve times)''. This will create a new Praat object, untitled. Use the rename button to rename it s, l, i or t to help you remember which phoneme it contains. 
\action Create a spectrum (not a spectogram) object for the s (fricative) and i (vowel) sound and compare the two
\action Now compare the s and i spectrums to the corresponding part of the spectogram for slit. 
\end{exercise}

Previously, we looked at harmonic frequencies in the bassoon sound.
Amplified harmonics in speech sounds show up as peaks in the spectrum,
or dark spots in the spectogram. These peaks are called formants. Vowels
can be differentiated by looking at how their formants are distributed.

\begin{exercise}
\ask How can you identify a fricative in the spectogram?
\askstar How can you recognize a plosive in the spectogram? And in the waveform?
\end{exercise}

\subsection{The sound of silence}\label{the-sound-of-silence}

Very small changes to the signal can sometimes have dramatic effects on
perception. For example, inserting a small period of silence (silent
interval) at specific places in words can create the effect of hearing
an extra phoneme. In this final part of the lab we'll explore the effect
of inserting a small silence in our recording of ``slit'' at just the
right place.

First, we'll create a small silence to be inserted into the
\file{slit.wav} sound. To find out an appropriate duration for this
silence, we'll look at a paper that investigated the effect of a silent
interval in the word ``slit''. Have a look at the methods section, as
well as the graph with results, in the paper by \cite{Marcus1978} that's
attached to this lab (\file{paper.pdf}). Use the graph summarizing their
results to find a good duration for the silent interval.

\begin{exercise}
\action In the Praat objects window, go to the menu ``New'' $>$ ``Sound'' and click ``Create sound from formula''
\action Change the value of the ``Name'' field to ``silence''. 
\action Adjust the end time to the duration of the silent interval that you found
\action In the formula field, type ``0'' (zero)
\action Click OK
\action Open the View \& Edit screen for your new sound
\action Select the entire sound (have a look a the Select menu if you run into issues)
\action Copy it, using ``Edit $>$ Copy selection to Sound clipboard'' or Ctrl-c
\end{exercise}

Now we're going to insert the silence into our recording of the word
``slit''.

\begin{exercise}
\action Go to the View \& Edit window for the sound \file{slit.wav}
\action Using the spectogram and waveform, find a spot in between the ``s'' and the ``l'' sound and place the cursor there
\end{exercise}

To prevent sudden jumps in the waveform, we should insert our silence at
a moment where the wave crosses the zero line.

\begin{exercise}
\action After having placed the cursor between the ``s'' and ``l'' sound, click on ``Select'' $>$ ``Move cursor to nearest zero crossing''
\action Now insert the silence we copied earlier by clicking ``Edit'' $>$ ``Paste after selection'', or by pressing Ctrl-v.
\action Play the sound. Which word do you hear?
\end{exercise}

\printbibliography

\end{document}
